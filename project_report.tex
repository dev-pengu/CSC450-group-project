\documentclass[12pt]{article}
\usepackage{url}
\usepackage{setspace}               
\usepackage[superscript]{cite}      
\usepackage{graphicx}               
\usepackage[normalem]{ulem}   		
\graphicspath{ {Figures/} }         
\usepackage{caption} 
\usepackage{cite}
\usepackage{indentfirst} 
\usepackage{float}
\usepackage{subcaption}
\usepackage{amsmath}  				
\textwidth=6.5in                    
\oddsidemargin=0.0in                
\usepackage{listings}
\usepackage{listings}
\usepackage{fancyhdr}
\usepackage{longtable}
\usepackage[table]{xcolor}
\pagestyle{fancy}
\fancyhf{}
\lhead{Family Organization }
\rhead{Page \thepage}

\usepackage{color}   
\usepackage{hyperref}
\hypersetup{
    colorlinks=true,
    citecolor=black,
    linktoc=all, 
    linkcolor=black,
}

\begin{document}

\begin{titlepage}

\newcommand{\HRule}{\rule{\linewidth}{0.5mm}} 

\center 

\textsc{\LARGE Missouri State University\\~\\Department computer Science}\\[1.0cm] 

\HRule \\[0.4cm]
{ \huge \bfseries Family Organization}\\[0.4cm] 
\HRule \\[1.5cm]



\begin{minipage}{0.4\textwidth}
\begin{flushleft} \large
Adam Barnes \\ Dala Biart \\ Kyle Chrzanowski \\ Matt Lippelman \\ 
\end{flushleft}
\end{minipage}
~
\begin{minipage}{0.4\textwidth}
\begin{flushright} \large
Mohammed Belkhouche \\ 
YassineBelkhouche@MissouriState.edu
\end{flushright}
\end{minipage}\\[2cm]

{\large \today}\\[2cm] 

\end{titlepage}

\newpage
%-----------------------------------------------------------------------
\tableofcontents

\newpage
%----------------------------------------------------------------------

%-----------------------------------------------------------------------------
\section {Family Organization}
\subsection{Software Description}
We plan to make a family organization and interaction application. It is a web application meant to keep track of different 
kinds of interactions a family might have, whether that be meal planning, creating a shopping list, or just keeping track of 
family events. Each functionality is meant to be customizable to fit the users' needs. We also plan to add authorization to offer 
security to the users, ensuring that important knowledge and data stays within the family. The application will be scale-able through 
additional modules that add different functionalities, as we don't yet know how big the project will become. Modules are listed and 
explained below, with modules we believe we can feasibly implement in the allotted time under section 2 and any 
additional proposed modules listed under section 3.

\section{Functions}
\subsection{Basic Functionality}
\begin{itemize}
    \item Basic authentication with email and password
    \item User will be able to create a family entity
    \item User will be able to add other users to their family
    \item User will be able to assign family roles to members of their family
    \item Users will only be able to access data for families they are part of
\end{itemize}
\subsection{Proposed Modules}
\begin{itemize}
    \item To-Do List: A To-Do list for each person in the family, to-dos will be colored to represent
            the different family members. Ability to create, read, update, and delete tasks will be included.
            Proposed module would have filters for viewing specific tasks or viewing other family member's tasks.
    \item Calendar: A calendar of all family events and individual events. We picture this module being particularly
            helpful in keeping track of the extra curricular activities that children have as well as planning
            family meetings, etc. Like the to-do list, the calendar events should be color coded per family member
            for a better user experience when viewing the calendar and filters should be included. The Calendar should have 
            the ability to create, read, update, and delete calendar events.
    \item Polling App: Members of a family can create polls for other family members to participate in. We see this being
            used for things like planning family events, picking a movie for a family night, picking out meals to eat for
            the week from a list of possibilities, etc. 
\end{itemize}

\begingroup
\section{Extra Functions (if we have extra time)}
\subsection{Proposed Modules}
\begin{itemize}
    \item Shopping List: A simple list of items to get while at the store. Members of the family should be able to add
            to the list, remove items from the list, and update items. This should be a list of items with the corresponding
            quantities.
    \item Meal Planner: For the Meal Planner module, the user would be presented with a 7 day calendar view where they can
            plan meals for the week. This could work with the shopping list with the ability to export the ingredients needed
            for each meal to the shopping list with appropriate quantities. We see users having the ability to add their own recipes
            as well as publish them to be used by other families.
    \item Family Chat: The Family Chat module would be a simple group chat feature, where families can have a conversation that 
            doesn't clog up their SMS inboxes as well as be available to family members without phone access.
    \item Shared File Uploads: With this module, family members would have the ability to upload files that others in the family
            would be able to access. It could be anything from a shared photo album to a vacation itenerary.
\end{itemize}
\end{document}
